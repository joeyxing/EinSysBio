\PassOptionsToPackage{unicode=true}{hyperref} % options for packages loaded elsewhere
\PassOptionsToPackage{hyphens}{url}
%
\documentclass[]{article}
\usepackage{lmodern}
\usepackage{amssymb,amsmath}
\usepackage{ifxetex,ifluatex}
\usepackage{fixltx2e} % provides \textsubscript
\ifnum 0\ifxetex 1\fi\ifluatex 1\fi=0 % if pdftex
  \usepackage[T1]{fontenc}
  \usepackage[utf8]{inputenc}
  \usepackage{textcomp} % provides euro and other symbols
\else % if luatex or xelatex
  \usepackage{unicode-math}
  \defaultfontfeatures{Ligatures=TeX,Scale=MatchLowercase}
\fi
% use upquote if available, for straight quotes in verbatim environments
\IfFileExists{upquote.sty}{\usepackage{upquote}}{}
% use microtype if available
\IfFileExists{microtype.sty}{%
\usepackage[]{microtype}
\UseMicrotypeSet[protrusion]{basicmath} % disable protrusion for tt fonts
}{}
\IfFileExists{parskip.sty}{%
\usepackage{parskip}
}{% else
\setlength{\parindent}{0pt}
\setlength{\parskip}{6pt plus 2pt minus 1pt}
}
\usepackage{hyperref}
\hypersetup{
            pdfborder={0 0 0},
            breaklinks=true}
\urlstyle{same}  % don't use monospace font for urls
\usepackage{color}
\usepackage{fancyvrb}
\newcommand{\VerbBar}{|}
\newcommand{\VERB}{\Verb[commandchars=\\\{\}]}
\DefineVerbatimEnvironment{Highlighting}{Verbatim}{commandchars=\\\{\}}
% Add ',fontsize=\small' for more characters per line
\newenvironment{Shaded}{}{}
\newcommand{\KeywordTok}[1]{\textcolor[rgb]{0.00,0.44,0.13}{\textbf{#1}}}
\newcommand{\DataTypeTok}[1]{\textcolor[rgb]{0.56,0.13,0.00}{#1}}
\newcommand{\DecValTok}[1]{\textcolor[rgb]{0.25,0.63,0.44}{#1}}
\newcommand{\BaseNTok}[1]{\textcolor[rgb]{0.25,0.63,0.44}{#1}}
\newcommand{\FloatTok}[1]{\textcolor[rgb]{0.25,0.63,0.44}{#1}}
\newcommand{\ConstantTok}[1]{\textcolor[rgb]{0.53,0.00,0.00}{#1}}
\newcommand{\CharTok}[1]{\textcolor[rgb]{0.25,0.44,0.63}{#1}}
\newcommand{\SpecialCharTok}[1]{\textcolor[rgb]{0.25,0.44,0.63}{#1}}
\newcommand{\StringTok}[1]{\textcolor[rgb]{0.25,0.44,0.63}{#1}}
\newcommand{\VerbatimStringTok}[1]{\textcolor[rgb]{0.25,0.44,0.63}{#1}}
\newcommand{\SpecialStringTok}[1]{\textcolor[rgb]{0.73,0.40,0.53}{#1}}
\newcommand{\ImportTok}[1]{#1}
\newcommand{\CommentTok}[1]{\textcolor[rgb]{0.38,0.63,0.69}{\textit{#1}}}
\newcommand{\DocumentationTok}[1]{\textcolor[rgb]{0.73,0.13,0.13}{\textit{#1}}}
\newcommand{\AnnotationTok}[1]{\textcolor[rgb]{0.38,0.63,0.69}{\textbf{\textit{#1}}}}
\newcommand{\CommentVarTok}[1]{\textcolor[rgb]{0.38,0.63,0.69}{\textbf{\textit{#1}}}}
\newcommand{\OtherTok}[1]{\textcolor[rgb]{0.00,0.44,0.13}{#1}}
\newcommand{\FunctionTok}[1]{\textcolor[rgb]{0.02,0.16,0.49}{#1}}
\newcommand{\VariableTok}[1]{\textcolor[rgb]{0.10,0.09,0.49}{#1}}
\newcommand{\ControlFlowTok}[1]{\textcolor[rgb]{0.00,0.44,0.13}{\textbf{#1}}}
\newcommand{\OperatorTok}[1]{\textcolor[rgb]{0.40,0.40,0.40}{#1}}
\newcommand{\BuiltInTok}[1]{#1}
\newcommand{\ExtensionTok}[1]{#1}
\newcommand{\PreprocessorTok}[1]{\textcolor[rgb]{0.74,0.48,0.00}{#1}}
\newcommand{\AttributeTok}[1]{\textcolor[rgb]{0.49,0.56,0.16}{#1}}
\newcommand{\RegionMarkerTok}[1]{#1}
\newcommand{\InformationTok}[1]{\textcolor[rgb]{0.38,0.63,0.69}{\textbf{\textit{#1}}}}
\newcommand{\WarningTok}[1]{\textcolor[rgb]{0.38,0.63,0.69}{\textbf{\textit{#1}}}}
\newcommand{\AlertTok}[1]{\textcolor[rgb]{1.00,0.00,0.00}{\textbf{#1}}}
\newcommand{\ErrorTok}[1]{\textcolor[rgb]{1.00,0.00,0.00}{\textbf{#1}}}
\newcommand{\NormalTok}[1]{#1}
\setlength{\emergencystretch}{3em}  % prevent overfull lines
\providecommand{\tightlist}{%
  \setlength{\itemsep}{0pt}\setlength{\parskip}{0pt}}
\setcounter{secnumdepth}{0}
% Redefines (sub)paragraphs to behave more like sections
\ifx\paragraph\undefined\else
\let\oldparagraph\paragraph
\renewcommand{\paragraph}[1]{\oldparagraph{#1}\mbox{}}
\fi
\ifx\subparagraph\undefined\else
\let\oldsubparagraph\subparagraph
\renewcommand{\subparagraph}[1]{\oldsubparagraph{#1}\mbox{}}
\fi

% set default figure placement to htbp
\makeatletter
\def\fps@figure{htbp}
\makeatother


\date{}

\begin{document}

\hypertarget{header-n3}{%
\section{System biology module 3 part A}\label{header-n3}}

\hypertarget{header-n5}{%
\subsection{General purpose functions}\label{header-n5}}

\hypertarget{header-n6}{%
\subsubsection{Trace back}\label{header-n6}}

Personally speaking, I really like this function \texttt{trace\_back}
because it uses recursion to solve the problem. Argument explanation is
available in the docstring.

The base condition of the recursion varies as the alignment method
changes.For global alignment, the base recursion is when \texttt{i} and
\texttt{j} are both \texttt{0},because according to definition, the
start of global alignment is on the top-left corner. For semi-global
alignment, the base recursion is when one of \texttt{i} and \texttt{j}
is \texttt{0} because the starting point of semi-global alignment is on
the first column or first row. For local alignment, it can be more
complicated because it can stop anywhere in the score matrix. According
to the algorithm of local alignment, when a subsequence starts, the
score at the start point should be \texttt{0} and none of the three
directions can yield a score of \texttt{0} ( see the last \texttt{if}
statement of function \texttt{trace\_back} ).

\texttt{trace\_back} returns a list \texttt{trace\_list} whose elements
are also lists in the following form:

\begin{Shaded}
\begin{Highlighting}[]
\NormalTok{[[s0, s1, s2, s3], [t1, t2, t3], (a, b)]}
\end{Highlighting}
\end{Shaded}

\texttt{trace\_list{[}0{]}} records all the position in sequecne
\texttt{s} where gap should be inserted. \texttt{trace\_list{[}1{]}}
records all the position in sequence \texttt{t} where gap should be
inserted. \texttt{(a,\ b)} is the start point of the sequence alignment
or end point of the recursion.

\hypertarget{header-n31}{%
\subsubsection{Print sequence}\label{header-n31}}

\texttt{print\_sequence} prints the result of the sequence in a more
beautiful and intuitive way. A match is represented by
\texttt{\textbar{}}. \texttt{-} means a gap and \texttt{x} means a
mismatch.

For example: local alignment

\begin{verbatim}
CTACGGGATCGTGATCGTAGCTAGGATGCTAGCAAGCTAGCACATCAGCAAACATCGACG
              |||||xxx||xxx||x|xxx|||||||xx|||xx||-x|||||||
           GCTTCGTACGCAGCTAGCCATTCAGCTAGCGGATCGACA-GCATCGACTAGGCAT
\end{verbatim}

\hypertarget{header-n39}{%
\subsubsection{align\_trace\_print}\label{header-n39}}

This function is nothing but a wrapper of the three alignment function,
\texttt{trace\_back} and \texttt{print\_sequence} to make the programme
more simple. It takes three arguments,

\texttt{s}: sequence1 \texttt{t}: sequence2 \texttt{alignment}: string
"global", "semi" or "local"

\hypertarget{header-n48}{%
\subsection{(a) Global alignment}\label{header-n48}}

Initialize a matrix, fill in the first row and first column then use
dynamic programming algorithm to fill in the rest entries.

Time complexity: \(O(len(s) \cdot len(t))\)

\hypertarget{header-n54}{%
\subsection{(b) Semi-global alignment}\label{header-n54}}

Similar to global alignment but fill the first row and first column with
zeros.

Time complexity: \(O(len(s) \cdot len(t))\)

\hypertarget{header-n59}{%
\subsection{(c) Local alignment}\label{header-n59}}

Time complexity: \$O(len(s) \textbackslash{}cdot len(t))

\hypertarget{header-n62}{%
\subsection{(d) special semi-global alignment}\label{header-n62}}

I use a matrix \texttt{A\_gap} of size
\texttt{{[}len(s),\ len(t),\ 2{]}} to record the number of gaps for both
sequences on a point.

\texttt{A\_gap{[}i,\ j,\ 0{]}} is the number of gaps for \texttt{s} to
reach point \texttt{A{[}i,\ j{]}}. Similarly,
\texttt{A\_gap{[}i,\ j,\ 1{]}} is number of gaps for \texttt{t}

No element in \texttt{A\_gap} is greater than 10:

\begin{Shaded}
\begin{Highlighting}[]
\BuiltInTok{print}\NormalTok{ np.where(A_gap[:,:,}\DecValTok{0}\NormalTok{] }\OperatorTok{>} \DecValTok{10}\NormalTok{)}
\BuiltInTok{print}\NormalTok{ np.where(A_gap[:,:,}\DecValTok{1}\NormalTok{] }\OperatorTok{>} \DecValTok{10}\NormalTok{)}

\NormalTok{(array([], dtype}\OperatorTok{=}\NormalTok{int64), array([], dtype}\OperatorTok{=}\NormalTok{int64))}
\NormalTok{(array([], dtype}\OperatorTok{=}\NormalTok{int64), array([], dtype}\OperatorTok{=}\NormalTok{int64))}
\end{Highlighting}
\end{Shaded}

Since this is semi-global alignment and the gaps at the begining of the
matrix don't have any cost, one cannot judge if the number of gaps at a
point is over 10 without calculating it. Hence, the time complexity of
the programme cannot be reduced to lower than
\(O(3 \cdot len(s) \cdot len(t)) =O(len(s) \cdot len(t))\)

\hypertarget{header-n78}{%
\subsection{Result of the whole script}\label{header-n78}}

\begin{verbatim}
System Biology Module 3 Part A

(a): Global Alignment
Score: 2
[[[], [2, 5], (0, 0)], [[], [3, 5], (0, 0)]]

Alignment0:
ACAAGGA
||-|||-
AC-AGG-

Alignment1:
ACAAGGA
|||-||-
ACA-GG-

(b): Semi-global Alignment
Score: 12
[[[], [], (2, 0)]]

Alignment0:
AGCCAATTACCAATTAAGG
  ||||||
  CCAATT
[[[], [], (9, 0)]]

Alignment1:
AGCCAATTACCAATTAAGG
         ||||||
         CCAATT

(c): Local Alignment
Score: 9

Alignment0:
AGCCTTCCTAGGG
   ||||x|
  GCTTCGTTT
(d):
18956
18957
(array([], dtype=int64), array([], dtype=int64))
(array([], dtype=int64), array([], dtype=int64))
37248
\end{verbatim}

\end{document}
