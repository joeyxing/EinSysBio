\PassOptionsToPackage{unicode=true}{hyperref} % options for packages loaded elsewhere
\PassOptionsToPackage{hyphens}{url}
%
\documentclass[]{article}
\usepackage{lmodern}
\usepackage{amssymb,amsmath}
\usepackage{ifxetex,ifluatex}
\usepackage{fixltx2e} % provides \textsubscript
\ifnum 0\ifxetex 1\fi\ifluatex 1\fi=0 % if pdftex
  \usepackage[T1]{fontenc}
  \usepackage[utf8]{inputenc}
  \usepackage{textcomp} % provides euro and other symbols
\else % if luatex or xelatex
  \usepackage{unicode-math}
  \defaultfontfeatures{Ligatures=TeX,Scale=MatchLowercase}
\fi
% use upquote if available, for straight quotes in verbatim environments
\IfFileExists{upquote.sty}{\usepackage{upquote}}{}
% use microtype if available
\IfFileExists{microtype.sty}{%
\usepackage[]{microtype}
\UseMicrotypeSet[protrusion]{basicmath} % disable protrusion for tt fonts
}{}
\IfFileExists{parskip.sty}{%
\usepackage{parskip}
}{% else
\setlength{\parindent}{0pt}
\setlength{\parskip}{6pt plus 2pt minus 1pt}
}
\usepackage{hyperref}
\hypersetup{
            pdfborder={0 0 0},
            breaklinks=true}
\urlstyle{same}  % don't use monospace font for urls
\usepackage{color}
\usepackage{fancyvrb}
\newcommand{\VerbBar}{|}
\newcommand{\VERB}{\Verb[commandchars=\\\{\}]}
\DefineVerbatimEnvironment{Highlighting}{Verbatim}{commandchars=\\\{\}}
% Add ',fontsize=\small' for more characters per line
\newenvironment{Shaded}{}{}
\newcommand{\KeywordTok}[1]{\textcolor[rgb]{0.00,0.44,0.13}{\textbf{#1}}}
\newcommand{\DataTypeTok}[1]{\textcolor[rgb]{0.56,0.13,0.00}{#1}}
\newcommand{\DecValTok}[1]{\textcolor[rgb]{0.25,0.63,0.44}{#1}}
\newcommand{\BaseNTok}[1]{\textcolor[rgb]{0.25,0.63,0.44}{#1}}
\newcommand{\FloatTok}[1]{\textcolor[rgb]{0.25,0.63,0.44}{#1}}
\newcommand{\ConstantTok}[1]{\textcolor[rgb]{0.53,0.00,0.00}{#1}}
\newcommand{\CharTok}[1]{\textcolor[rgb]{0.25,0.44,0.63}{#1}}
\newcommand{\SpecialCharTok}[1]{\textcolor[rgb]{0.25,0.44,0.63}{#1}}
\newcommand{\StringTok}[1]{\textcolor[rgb]{0.25,0.44,0.63}{#1}}
\newcommand{\VerbatimStringTok}[1]{\textcolor[rgb]{0.25,0.44,0.63}{#1}}
\newcommand{\SpecialStringTok}[1]{\textcolor[rgb]{0.73,0.40,0.53}{#1}}
\newcommand{\ImportTok}[1]{#1}
\newcommand{\CommentTok}[1]{\textcolor[rgb]{0.38,0.63,0.69}{\textit{#1}}}
\newcommand{\DocumentationTok}[1]{\textcolor[rgb]{0.73,0.13,0.13}{\textit{#1}}}
\newcommand{\AnnotationTok}[1]{\textcolor[rgb]{0.38,0.63,0.69}{\textbf{\textit{#1}}}}
\newcommand{\CommentVarTok}[1]{\textcolor[rgb]{0.38,0.63,0.69}{\textbf{\textit{#1}}}}
\newcommand{\OtherTok}[1]{\textcolor[rgb]{0.00,0.44,0.13}{#1}}
\newcommand{\FunctionTok}[1]{\textcolor[rgb]{0.02,0.16,0.49}{#1}}
\newcommand{\VariableTok}[1]{\textcolor[rgb]{0.10,0.09,0.49}{#1}}
\newcommand{\ControlFlowTok}[1]{\textcolor[rgb]{0.00,0.44,0.13}{\textbf{#1}}}
\newcommand{\OperatorTok}[1]{\textcolor[rgb]{0.40,0.40,0.40}{#1}}
\newcommand{\BuiltInTok}[1]{#1}
\newcommand{\ExtensionTok}[1]{#1}
\newcommand{\PreprocessorTok}[1]{\textcolor[rgb]{0.74,0.48,0.00}{#1}}
\newcommand{\AttributeTok}[1]{\textcolor[rgb]{0.49,0.56,0.16}{#1}}
\newcommand{\RegionMarkerTok}[1]{#1}
\newcommand{\InformationTok}[1]{\textcolor[rgb]{0.38,0.63,0.69}{\textbf{\textit{#1}}}}
\newcommand{\WarningTok}[1]{\textcolor[rgb]{0.38,0.63,0.69}{\textbf{\textit{#1}}}}
\newcommand{\AlertTok}[1]{\textcolor[rgb]{1.00,0.00,0.00}{\textbf{#1}}}
\newcommand{\ErrorTok}[1]{\textcolor[rgb]{1.00,0.00,0.00}{\textbf{#1}}}
\newcommand{\NormalTok}[1]{#1}
\setlength{\emergencystretch}{3em}  % prevent overfull lines
\providecommand{\tightlist}{%
  \setlength{\itemsep}{0pt}\setlength{\parskip}{0pt}}
\setcounter{secnumdepth}{0}
% Redefines (sub)paragraphs to behave more like sections
\ifx\paragraph\undefined\else
\let\oldparagraph\paragraph
\renewcommand{\paragraph}[1]{\oldparagraph{#1}\mbox{}}
\fi
\ifx\subparagraph\undefined\else
\let\oldsubparagraph\subparagraph
\renewcommand{\subparagraph}[1]{\oldsubparagraph{#1}\mbox{}}
\fi

% set default figure placement to htbp
\makeatletter
\def\fps@figure{htbp}
\makeatother


\date{}

\begin{document}

\hypertarget{header-n0}{%
\section{System biology module 3 part B}\label{header-n0}}

\hypertarget{header-n2}{%
\subsection{File description}\label{header-n2}}

\hypertarget{header-n5}{%
\subsection{General Purpose Modules}\label{header-n5}}

usage:

\begin{Shaded}
\begin{Highlighting}[]
\ImportTok{from}\NormalTok{ draw_MR }\ImportTok{import} \OperatorTok{*}
\ImportTok{from}\NormalTok{ degreeAnalysis }\ImportTok{import} \OperatorTok{*}
\end{Highlighting}
\end{Shaded}

\hypertarget{header-n9}{%
\subsubsection{draw\_MR.py}\label{header-n9}}

\texttt{read\_MR\_txt}: read from a txt file, for example,
fructoseanaerobicfluxgraph.txt and return the list of edges \texttt{el}
and nodes \texttt{nl} in the graph.

\texttt{draw\_MR\_Graph}: use the list returned by
\texttt{read\_MR\_txt} to generate a \texttt{networkx} graph.

\hypertarget{header-n16}{%
\subsubsection{degreeAnalysis.py}\label{header-n16}}

\texttt{my\_dijkstra}: calculate the shortest distance from a source
node to the rest of the nodes using Dijkstra algorithm. \texttt{weight}
is a optional argument and it can be \texttt{string}, \texttt{int} or
\texttt{float}.

\begin{itemize}
\item
  \texttt{string}: use the value in the field of the string as the
  weight.
\item
  \texttt{int} or \texttt{float}: use this value as the fixed weight for
  all edges.
\end{itemize}

\texttt{degree\_stats}: plot a histogram of the number of the nodes with
different numbers of degree. Print the average degree, variance of
degree and the hubs node (with greatest degree).

\texttt{draw\_bar}: draw a bar chart using the given data. It is called
in \texttt{degree\_stats} and \texttt{compartment\_analysis}.

\texttt{compartment\_analysis}: plot a histogram of the number of
reactions and metabolites with regard to the compartment they belong to.

\hypertarget{header-n38}{%
\subsection{B.2, B.3}\label{header-n38}}

Please refer to notebook \texttt{Exercise\_B2B3.py} or report
\texttt{Exercise\_B2B3.html}.

\hypertarget{header-n41}{%
\subsection{B.4}\label{header-n41}}

To block the product \texttt{M\_mthgxl\_e}, one has to block all the
reactions that produces it. There is only one reaction in the graph that
produce \texttt{M\_mthgxl\_e}, which is
\texttt{R\_EX\_mthgxl(e),\ ID\ =\ R\_EX\_mthgxl\_LPAREN\_e\_RPAREN\_}.

I did the \textbf{following} work in this script:

\begin{enumerate}
\def\labelenumi{\arabic{enumi}.}
\item
  add two functions \texttt{read\_block\_list} and
  \texttt{block\_reactions}.

  \begin{itemize}
  \item
    read\_block\_list: read the reactions to block from a text file and
    return the reactions as a list
  \item
    block\_reactions: set the kineticLaw parameters
    \texttt{UPPER\_BOUND} and \texttt{LOWER\_BOUND} of all the reactions
    in block list to zeros so that the reaction is blocked from both
    sides
  \end{itemize}
\item
  modify your function \texttt{max\_flux} to \texttt{my\_max\_flux} so
  that it calls my functions described above when calculating the
  maximum flux. Also modify \texttt{max\_fluxes} to
  \texttt{my\_max\_fluxes} in which \texttt{max\_flux} is called.
\end{enumerate}

The resulted R/M graph can be plotted using script \texttt{draw\_MR.py}:

\begin{Shaded}
\begin{Highlighting}[]
\NormalTok{$ }\ExtensionTok{python}\NormalTok{ draw_MR.py exclude_toxic.txt}
\end{Highlighting}
\end{Shaded}

Result of script is stored in file \texttt{exclude\_toxic.txt}. It can
be verified that there are no product \texttt{M\_mthgxl\_b} any more.

\hypertarget{header-n73}{%
\subsection{B.5, B.6}\label{header-n73}}

First, in script Exercise\_B5B6.py, I generate the text file of
different graphs.

\begin{itemize}
\item
  Fructose aerobic: EX\_fru(e)True.txt
\item
  Glucose anaerobic: EX\_glc(e)False.txt
\item
  Glucose aerobic: EX\_glc(e)True.txt
\item
  Fructose + glucose anaerobic: EX\_glc(e)EX\_fru(e)False.txt
\item
  Fructose + glucose aerobic: EX\_glc(e)EX\_fru(e)True.txt
\end{itemize}

\hypertarget{header-n92}{%
\subsubsection{Flux result:}\label{header-n92}}

\begin{itemize}
\item
  Fructose aerobic: 1600
\item
  Glucose anaerobic: 100
\item
  Glucose aerobic: 1600
\item
  Fructose + glucose anaerobic: 100
\item
  Fructose + glucose aerobic: 1600
\end{itemize}

\hypertarget{header-n109}{%
\subsubsection{Plotting}\label{header-n109}}

Then, in notebook Exercise\_B5B6.ipynb, I repeated all the analysis for
these new graphs. Please read the notebook or report Exercise\_B5B6.html
for more details.

\end{document}
